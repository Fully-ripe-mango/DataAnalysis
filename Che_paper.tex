\documentclass{article}
% Packages for formatting and customization
\usepackage[style=numeric,sorting=nyt]{biblatex}  % 选择APA风格,其他风格也可选择
\addbibresource{my.bib}
\usepackage{amsmath}
\usepackage{float}

\usepackage{ctex}
\usepackage{amssymb}
\usepackage{graphicx}
\usepackage{hyperref}
\usepackage{geometry}
\usepackage{subcaption}
\usepackage{multicol}  % For splitting the page into columns
\usepackage{titlesec}   % For customizing section titles

\geometry{a4paper, margin=1in}
\usepackage{fancyhdr}
\pagestyle{fancy}
\fancyhead[L]{\textit{化学}}  % Left header
\fancyhead[R]{\textit{速干衣原理机制分析}}  % Right header
\titlespacing*{\section}{0pt}{1ex plus .2ex minus .2ex}{1ex plus .2ex}  % Adjust spacing for section titles
\titlespacing*{\subsection}{0pt}{1ex plus .2ex minus .2ex}{0.5ex plus .1ex}  % Adjust spacing for subsection titles

% Title and Author information

\begin{document}

\begin{center}
    {\LARGE \textbf{速干衣原理机制分析}} \\[1em]
    \vspace{0.1em}  % 调整标题与横线之间的间距
    \hrule
    \begin{flushright}  % 使副标题靠右
        \textit{\small 化学小组大作业}  % 副标题样式
    \end{flushright}
    {\large 崔武扬\quad 田渝翔\quad  肖宇\quad  张静轩} \\[0.5em]
    {\large \today}
\end{center}

\begin{abstract}
    速干衣通过采用特殊材料和设计,能够迅速将汗水和水分从皮肤表面传导到外部,保持穿着者干爽舒适。常见的速干衣材料包括聚酯纤维、尼龙和氨纶,每种材料具有不同的速干性能和舒适性。速干衣的关键原理包括纤维的微观结构、吸湿机理和水分传输过程。通过改性技术提高材料的吸湿性和速干性,织物的设计则利用毛细管效应和微孔薄膜技术加速水分蒸发。本文通过对速干衣的工作原理和材料特性的研究,期望增进对速干衣材料理解、优化生产技术和提高消费者体验。
\end{abstract}

\begin{multicols}{2}

\section{引言}
\subsection{背景介绍}
速干衣,也称为快干衣,是一种利采用特殊材料和特殊设计的服装,其主要功能是能够快速将汗水和水分从皮肤表面传导到衣物外部,从而保持穿着者的干爽舒适。这种衣物在市场上广泛应用于户外运动、健身、旅行等多种场合,因其能够迅速干燥,减少因潮湿衣物引起的不适感和健康风险,在日常生活中越来越成为人们的穿衣选择。
\subsection{研究目的}
通过了解速干衣的工作原理,我们可以更好地理解其内部所包含的化学知识和材料的技术特点,并评估其在实际应用中的效果和性能。
\subsection{研究意义}
材料科学:了解速干衣的材料特性和化学原理有助于从其中发散创新型思维,举一反三开发新型高效能材料,推动材料科学的进步。

纺织企业:掌握速干衣的制造技术可以优化生产流程,提高产品质量,占据高端技术衣物市场。

消费者选择:消费者通过了解速干衣的原理,可以明白各种材料的舒适性与速干性,从而更加明智地选择适合自己需求的产品,提高使用体验。
\section{速干衣材料基础}
\subsection{主要材质}
速干衣中常用的材质包括聚酯纤维、尼龙和氨纶等。这些材料具有不同的特性:
聚酯纤维:具有很好的耐磨性和抗皱性,同时能够快速干燥,是速干衣常用的材料之一。
尼龙:轻质、柔软,具有良好的弹性和耐磨性,但透气性相对较差。
氨纶:也称为弹性纤维,能够提供衣物的弹性,使穿着更加舒适。
\subsection{材质选择}
不同材质在速干性能方面的优缺点如下:
聚酯纤维:优点是快干、耐磨;缺点是透气性相对较差。
尼龙:优点是轻质、柔软;缺点是透气性差,干燥速度慢。
氨纶:优点是提供良好的弹性和舒适度;缺点是单独使用时不具备速干功能。
材质选择对速干衣性能的影响主要体现在:
透气性:影响穿着者的舒适度和健康。
快干性:影响衣物在潮湿环境下的干燥速度。
耐磨性:影响衣物的耐用性和使用寿命。
弹性:影响穿着的舒适度和活动自由度


\begin{figure}[H]
    \centering
    \includegraphics[width=0.5\textwidth]{pic2.jpg}
    \caption{TOPCOOL微观图片}
    \label{fig:image_label}
\end{figure}


\section{纤维微观结构和吸湿机理\cite{Wei2015}} 
\subsection{纤维微观结构\cite{Zhu2006}}
首先介绍速干衣常用纤维的微观结构。聚酯纤维是由大分子链中的各链节通过酯基连接而成。聚酯纤维的分子结构是由对苯二甲酸和乙二醇聚合而成,分子量通常在18000到25000之间。这种线型或支化度很低的分子结构,以及超分子结构中的取向和部分结晶,赋予了聚酯纤维特定的耐热性、机械物理性能和化学稳定性。氨纶的分子链由软链段和硬链段组成。软链段是不具结晶性、分子量较大的低熔点聚酯或聚醚长链,它们组成纤维的无定形区段,分子量在2000~4000之间,在常温下处于高弹态,分子链卷曲,应力作用下很容易产生形变,纤维容易被拉长。硬链段由能形成氢键、易生成结晶结构或能产生横向交联的芳香族二异氰酸酯和链增长剂组成,具有高度对称性。 

\begin{figure}[H]
    \centering
    \includegraphics[width=0.5\textwidth]{pic1.jpg}
    \caption{TOPCOOL微观图片}
    \label{fig:image_label}
\end{figure}
\subsection{吸湿机理\cite{Yu2020} \cite{zhang2013}} 
纤维的吸湿性主要取决于其中的亲水基团以及其本身的纤维微观结构。速干衣主流材料中的亲水基团主要包括羧基(—COOH)、酰氨基(—CONH2)、羟基(—OH)和氨基(—NH2)。这些基团对水分子有较强的亲和力,与水分子缔合形成氢键,这是一种介于范德华力和化学键之间的作用力,使水分子失去热运动的能力,从而在纤维中暂时停留。纤维结构通过横截面形状、表面积和沟槽数量影响吸湿导湿功能。例如,异形截面纤维通过改变纤维的横截面形状,增加了纤维的表面积和沟槽数量,从而提高了纤维的吸湿导湿性能;“十”字形截面的coolmax纤维具有四条纵向沟槽,比普通纤维多出20的表面积,这使得水分能够在纤维内部快速迁移和扩散。


需要注意到,一些纤维如聚酯纤维的天然亲水性并不好,需要通过物理、化学方法对其进行亲水——速干改性,目前主流的方法大概有五种,分别是共聚法、接枝改性法、表面处理法、共混法和物理方法。共聚法是通过将亲水性单体与纤维大分子共聚,使亲水基团直接嵌入到纤维的化学结构中。接枝改性法是在纤维成型后,利用放射线或其他化学方法使亲水性单体(如丙烯酸、无水顺丁烯二酸)接枝到纤维表面,从而提高纤维的亲水性和耐热性。表面处理法则是通过化学处理或物理处理使纤维表面具有高比表面积,并引入亲水基团。共混法是将含有亲水基团的聚合物与其他聚合物共混,形成复合纤维,从而有效提高纤维的吸湿性和速干性能。物理方法如使纤维具有多孔结构、表面粗糙化及横截面异形化等,这些方法虽然不直接引入亲水基团,但可以通过增加纤维的比表面积和毛细管效应来增强其吸湿和速干性能。
\section{速干衣水分传输机制与结构设计}
\subsection{水分传输和蒸发过程\cite{Hu2021}} 
接下来介绍速干衣水分传输与蒸发的过程。首先,汗液接触纤维和织物表面,通过润湿作用使水分与纤维接触。水分浸润纤维和织物表面后,进入织物内部结构完成吸湿过程。此时,水分被纤维和织物吸收,并在内部逐渐扩散。之后,水分从织物内表面向外表面的运输主要依靠纤维内空腔、单纤维内的孔洞、纱线内单纤维间的缝隙以及织物中纱线间的缝隙所形成的毛细管效应。根据毛细流动理论,当液体与基质表面分子之间的附着力大于液体本身分子间的凝聚力时,就会发生毛细现象,即液体在多孔材料中反重力方向的移动,所以在设计速干衣时,通常材料的内层到外层所用的纤维原料具有不同的吸湿性能,内层疏水,外层亲水,纤维形成的毛细管从内层到外层也呈由粗到细的变化。这种设计使得毛细管芯吸作用随着毛细管半径减小而逐渐增大,从而在材料的内外层界面产生附加压力差。毛细管曲面的附加压力作用及液体表面的张力作用会引导液体自动从内层传流到外层。最终,吸收的水分通过织物表面向外层空间蒸发,从而实现速干效果。这一过程受外界环境的温度、湿度和空气流速的影响较大。
\subsection{织物设计结构\cite{Chen2005}}

我们已经知道了速干衣水分运输和蒸发的原理,那么速干衣是通过怎样的织物结构设计来加速这一过程的呢?首先是微孔薄膜技术,微孔薄膜通常由聚四氟乙烯(PTFE)或聚偏氟乙烯(PVDF)等高分子材料制成。这些材料在低于熔点下加热制成薄膜,然后通过急速拉伸冷却形成具有热固型蛛网状微细多孔结构的薄膜。每平方英寸的薄膜上可能有超过$10^9$个微孔,这些微孔的尺寸介于雾滴与水蒸气分子之间,有效防止水滴进入织物,同时允许汗汽通过。同时,速干衣织物的组织结构、密度厚度和交织频率也是影响其吸湿导湿性能的关键因素,平纹组织和高密度的织物通常不利于水分的快速传输,而透孔组织和适当的密度则有助于提高吸湿快干性能

\section{结论}
本篇论文通过对速干衣的工作原理和材料特性的深入分析,揭示了速干衣能够迅速将汗水和水分从皮肤表面传导到外部的关键因素。我们研究发现,速干衣的纤维微观结构、吸湿机理以及水分传输机制是实现速干效果的核心。通过改性技术提高材料的吸湿性和速干性,以及利用毛细管效应和微孔薄膜技术加速水分蒸发,达到了速干衣保持穿着者干爽舒适的效果。此外,织物的设计结构,如微孔薄膜技术和组织结构,对提高速干衣的吸湿导湿性能同样至关重要。随着材料科学和纺织技术的不断进步,速干衣的性能和应用范围有望进一步扩展,为穿着者提供更多的便利和舒适。
\section{展望}
综合这些研究,我们期望我们在这一次的研究中能够增进对速干衣材料的理解,同时对其中的化学与材料知识有了进一步的挖掘,在查找资料并整合的过程中,培养自身提出问题、分析问题、解决问题的能力。

\end{multicols}


\printbibliography
\section{分工}
本文由张静轩同学负责框架构建,肖宇、崔武扬同学负责文案编写,田渝翔同学负责论文排版
,贡献占比均为25\%

查看本论文LaTeX生成源码地址请点击此处\href{https://github.com/Fully-ripe-mango/Mycode/blob/main/Che_paper.tex}{此处}
\end{document}
